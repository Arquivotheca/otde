\documentclass{article}
\usepackage{linuxdoc-sgml}
\usepackage{qwertz}
\usepackage[latin1]{inputenc}
\usepackage{epsfig}
\usepackage{null}
\title{The KEdit Handbook}
\author{Bernd Johannes Wuebben, {\ttfamily wuebben@kde.org}}
\date{Version 0.5, 20 May 1997}
\abstract{This Handbook describes KEdit Version 0.5
}


\begin{document}
\maketitle
\tableofcontents

\section{Introduction}

KEdit is the default text editor for the KDE Desktop. It is a small editor best used
in conjunction with kfm for text and configuration file browsing. KEdit may also serve
well for composition of small plain text documents. It is not meant
to be a programmers editor, in particular it is not meant to replace any of the
more powerful editors such as XEmacs or Emacs. KEdit's functionality will intentionally
remain rather limited to ensure that KEdit will start  up reasonably fast.

I hope you will enjoy this editor,

{\ttfamily Bernd Johannes Wuebben}

wuebben@kde.org






\section{Installation}




\subsection{How to obtain KEdit}

KEdit is a core application of the KDE project \url{http://www.kde.org}{}.
KEdit can be found on \url{ftp://ftp.kde.org/pub/kde/}{}, the main ftp site
of the KDE project.




\subsection{Requirements}

In order to successfully compile KEdit, you need the latest versions of {\ttfamily  libkdecore}
and {\ttfamily  libkfm}. All required libraries as well as KEdit itself can be found
on \url{ftp://ftp.kde.org/pub/kde/}{}. 








\subsection{Compilation and installation}



In order to compile and install KEdit on your system, type the following in the base 
directory of the KEdit distribution:
\begin{tscreen}
\begin{verbatim}
% ./configure
% make
% make install
\end{verbatim}
\end{tscreen}




Since KEdit uses {\ttfamily autoconf} you should have not trouble compiling it.
Should you run into problems please report them to the {\sffamily KDE} mailing lists.




\section{Onscreen Fundamentals}



KEdit is very simply to use. I am sure that if you have ever used a text edit you 
will have no problems with KEdit. 






\subsection{Editing files on the internet}

You can open and save files transparently on the internet. Try the the following at the
command-line prompt to see an example of this.

\begin{tscreen}
\begin{verbatim}
% kedit ftp://ftp.kde.org/pub/kde/Welcome.msg
\end{verbatim}
\end{tscreen}





\subsection{Key Bindings}

KEdit honors the following key bindings.

\begin{itemize}
\item {\ttfamily Left Arrow} Move the cursor one character leftwards 
\item {\ttfamily Right Arrow} Move the cursor one character rightwards 
\item {\ttfamily Up Arrow} Move the cursor one line upwards 
\item {\ttfamily Down Arrow} Move the cursor one line downwards 
\item {\ttfamily Page Up} Move the cursor one page upwards 
\item {\ttfamily Page Down} Move the cursor one page downwards 
\item {\ttfamily Backspace} Delete the character to the left of the cursor 
\item {\ttfamily Home} Move the cursor to the beginning of the line 
\item {\ttfamily End} Move the cursor to the end of the line 
\item {\ttfamily Delete} Delete the character to the right of the cursor 
\item {\ttfamily Shift - Left Arrow} Mark text one character leftwards 
\item {\ttfamily Shift - Right Arrow} Mark text one character rightwards 
\item {\ttfamily Control-A} Move the cursor to the beginning of the line 
\item {\ttfamily Control-B} Move the cursor one character leftwards 
\item {\ttfamily Control-C} Copy the marked text to the clipboard. 
\item {\ttfamily Control-D} Delete the character to the right of the cursor 
\item {\ttfamily Control-E} Move the cursor to the end of the line 
\item {\ttfamily Control-F} Move the cursor one character rightwards 
\item {\ttfamily Control-H} Delete the character to the left of the cursor 
\item {\ttfamily Control-K} Delete to end of line 
\item {\ttfamily Control-N} Move the cursor one line downwards 
\item {\ttfamily Control-P} Move the cursor one line upwards 
\item {\ttfamily Control-V} Paste the clipboard text into line edit. 
\item {\ttfamily Control-X} Cut the marked text, copy to clipboard. 
\end{itemize}





\section{The Menu Entries}




\subsection{File}




\subsubsection{New}

Opens a new Document in the editor


\subsubsection{Open}

Allows the user to open a document


\subsubsection{Save}

Saves the current Document


\subsubsection{Save As}

Allows the user to save the document in a new file.


\subsubsection{Close}

Closes the editor window. If the closed editor window was the last instance open,
KEdit will exit.


\subsubsection{Open URL}

Allows the user to open a file on the internet.


\subsubsection{Save to URL}

Allows the user to save to a file on the internet.


\subsubsection{Print}

Open the print dialog and lets the user print the whole document or the current selection.


\subsubsection{New Window}




\subsubsection{Exit
Exits the Editor. }








\subsection{Edit}




\subsubsection{Copy }

Copies the current selection to the clip board.


\subsubsection{Cut}

Deletes the current selection and places it into the clip board


\subsubsection{Paste}

Inserts the content of the clip board at the current cursor position


\subsubsection{Insert File }

Allows the user to insert a file at the current cursor position.


\subsubsection{Insert Date}

Inserts the current date at the cursor position.


\subsubsection{Find}

Opens the find dialog. 


\subsubsection{Find Again }

Repeats the last find operation , if a find operation has already taken place.


\subsubsection{Replace}

Opens the replace dialog


\subsubsection{Goto Line}

Opens the goto line dialog




\subsection{Options}




\subsubsection{Font}

Allows the user to choose the font which the editor uses to display the text.
The font information is not stored with the document, nor can you choose different
fonts within a document. 


\subsubsection{Colors}

Allows the user to select fore and background color of KEdit's text display area.


\subsubsection{Fill Column}

Allows the user to set a fill column as well as to enable word wrap.


\subsubsection{Auto Indent}

Allows the user to switch to auto indent mode and back. In auto indent mode the 
cursor is placed on a  RETURN or ENTER underneath the first visible character 
on the first nonempty line above the current line.


\subsubsection{Hide/Show Tool-Bar}

Allows the user to have the tool bar displayed or hidden.


\subsubsection{Hide/Show Status-Bar}

Allows the user to have the status bar displayed or hidden.




\subsection{Help}




\subsubsection{Help}

Invokes the KDE help system with the KEdit help pages displayed.


\subsubsection{About }

Displays essential information about KEdit.




\section{Questions and answers}



\end{document}
