\documentclass{article}
\usepackage{linuxdoc-sgml}
\usepackage{palatino}
\usepackage{qwertz}
\usepackage[latin1]{inputenc}
\usepackage{epsfig}
\usepackage{null}
\title{The KFontManger Handbook}
\author{Bernd Johannes Wuebben, {\ttfamily wuebben@kde.org}}
\date{Version 0.2.2, 20 May 1997}
\abstract{This Handbook describes KFontManger Version 0.2.2
}


\begin{document}
\maketitle
\tableofcontents

\section{Introduction}



KFontManager is a font manager for the KDE project. KFontManger was designed
to work in conjunction with kfontdialog version 0.5 and higher.

The neccessity for a Font-Manager arose from the following facts:

\begin{itemize}
\item The font-list in kfontdialog used to be hard-coded. This was clearly
not desirable. 
\item Not all fonts that come with the X11 window system are useful. Were
kfontdialog simply to display all fonts availabe on your X server system,
you would over and over again be confronted with the same list of fonts,
some of which completely unusable. KFontManger allows you to select which
fonts you want to be available to KDE applications and in particular to
kfontdialog.
\end{itemize}





\section{Installation}



You need libkdecore and libqt1.2 in order to be able to compile KFontManager.
To compile and install KFontManger simply execute the following command
at the command-line prompt.

\begin{tscreen}
\begin{verbatim}
% ./configure
% make
% make install
\end{verbatim}
\end{tscreen}







\section{How it works}

KFontManager will present to you a list of all available fonts on you
system. You will then use kfontmanger to compose a list of KDE fonts
by choosing those fonts availabe on you system that you like. This list
of fonts will then be available to the KDE applications. For example kfontdialog,
which is used for example in kedit, will then offer exactly those fonts
you selected. The list of fonts you create using KFontManager will be stored
in the file ~/.kde/config/kdefonts. Unless you know what you are doing
you should refrain from manipulating the KDE font file by hand.


\section{For Developers}



A Kapplication method: getKDEFonts(QStrList~*fontlist) is availabe 
in KApplication. This method will enable you to obtain the
list of KDE fonts for use in your application, should you have a need for
it. Normally only kfontdialog would do this. 



{\ttfamily Bernd Johannes Wuebben}

wuebben@kde.org



\end{document}
